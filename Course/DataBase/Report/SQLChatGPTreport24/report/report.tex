\documentclass{article}


% Title
\title{Database Report}
\author{XIONG Zhipeng \  Sid: 44231536}
\date{\today}

\begin{document}

\maketitle

\section*{Question 1}
For each of the above query descriptions Q1, Q2, Q3, generate SQL queries by
using ChatGPT. Show the generated queries.

\begin{itemize}
    \item Q1: Find mid of menuitems such that the number of guests who ordered the menuitem
    only once is maximum. Report such mid and the maximum number of guests.
    \begin{verbatim}
        WITH GuestOrderCount AS (
            SELECT o.mid, o.gid, COUNT(*) AS order_count
            FROM Order o
            GROUP BY o.mid, o.gid
        ),
        MenuItemGuestCount AS (
            SELECT mid, COUNT(gid) AS guest_count
            FROM GuestOrderCount
            WHERE order_count = 1
            GROUP BY mid
        )
        SELECT mid, guest_count
        FROM MenuItemGuestCount
        ORDER BY guest_count DESC
        LIMIT 1;        
    \end{verbatim}
    \item Q2:  Find guests who always order an item that is most expensive within the table.
    \begin{verbatim}
        -- Step 1: Identify the most expensive menu item
        WITH MostExpensiveItem AS (
            SELECT mid, price
            FROM MenuItem
            WHERE price = (SELECT MAX(price) FROM MenuItem)
        ),

        -- Step 2: Find guests who have only ordered the most expensive item
        GuestOrders AS (
            SELECT gid
            FROM Order
            WHERE mid IN (SELECT mid FROM MostExpensiveItem)
            GROUP BY gid
            HAVING COUNT(DISTINCT mid) = (SELECT COUNT(*) FROM MostExpensiveItem)
        ),

        -- Step 3: Ensure these guests have not ordered any other items
        GuestValid AS (
            SELECT go.gid
            FROM GuestOrders go
            LEFT JOIN Order o ON go.gid = o.gid
            LEFT JOIN MostExpensiveItem mei ON o.mid = mei.mid
            GROUP BY go.gid
            HAVING COUNT(DISTINCT o.mid) = COUNT(DISTINCT mei.mid)
        )

        -- Step 4: Select the guests who meet the criteria
        SELECT g.gid, g.tableid
        FROM Guest g
        JOIN GuestValid gv ON g.gid = gv.gid;
    \end{verbatim}
    \item Q3: Find the golden guest pair such that their total amount in two visits is maximum
    than any other guest pairs. Guest pairs are any two guests coming on the same day
    and occupy one table by the two.
    \begin{verbatim}
        WITH GuestVisitAmount AS (
            SELECT
                g1.gid AS gid1,
                g2.gid AS gid2,
                g1.tableid,
                o1.date,
                SUM(o1.quantity * m1.price) AS amount1,
                SUM(o2.quantity * m2.price) AS amount2,
                SUM(o1.quantity * m1.price) + SUM(o2.quantity * m2.price) AS total_amount
            FROM
                Guest g1
                JOIN Guest g2 ON g1.tableid = g2.tableid AND g1.gid < g2.gid
                JOIN Order o1 ON g1.gid = o1.gid
                JOIN Order o2 ON g2.gid = o2.gid AND o1.date = o2.date
                JOIN MenuItem m1 ON o1.mid = m1.mid
                JOIN MenuItem m2 ON o2.mid = m2.mid
            GROUP BY
                g1.gid, g2.gid, g1.tableid, o1.date
        ),

        MaxTotalAmount AS (
            SELECT
                gid1, gid2, total_amount
            FROM
                GuestVisitAmount
            ORDER BY
                total_amount DESC
            LIMIT 1
        )

        SELECT
            mta.gid1, mta.gid2, mta.total_amount
        FROM
            MaxTotalAmount mta;
    \end{verbatim}
\end{itemize}

\end{document}