\documentclass{article}


% Title
\title{Database Report}
\author{XIONG Zhipeng \  Sid: 44231536}
\date{\today}

\begin{document}

\maketitle

\section*{Question 1}
For each of the above query descriptions Q1, Q2, Q3, generate SQL queries by
using ChatGPT. Show the generated queries.

\begin{itemize}
    \item Q1: Find mid of menuitems such that the number of guests who ordered the menuitem
    only once is maximum. Report such mid and the maximum number of guests.
    \begin{verbatim}
        WITH GuestOrderCount AS (
            SELECT o.mid, o.gid, COUNT(*) AS order_count
            FROM Order o
            GROUP BY o.mid, o.gid
        ),
        MenuItemGuestCount AS (
            SELECT mid, COUNT(gid) AS guest_count
            FROM GuestOrderCount
            WHERE order_count = 1
            GROUP BY mid
        )
        SELECT mid, guest_count
        FROM MenuItemGuestCount
        ORDER BY guest_count DESC
        LIMIT 1;        
    \end{verbatim}
    \item Q2:  Find guests who always order an item that is most expensive within the table.
    \begin{verbatim}
        -- Step 1: Identify the most expensive menu item
        WITH MostExpensiveItem AS (
            SELECT mid, price
            FROM MenuItem
            WHERE price = (SELECT MAX(price) FROM MenuItem)
        ),

        -- Step 2: Find guests who have only ordered the most expensive item
        GuestOrders AS (
            SELECT gid
            FROM Order
            WHERE mid IN (SELECT mid FROM MostExpensiveItem)
            GROUP BY gid
            HAVING COUNT(DISTINCT mid) = (SELECT COUNT(*) FROM MostExpensiveItem)
        ),

        -- Step 3: Ensure these guests have not ordered any other items
        GuestValid AS (
            SELECT go.gid
            FROM GuestOrders go
            LEFT JOIN Order o ON go.gid = o.gid
            LEFT JOIN MostExpensiveItem mei ON o.mid = mei.mid
            GROUP BY go.gid
            HAVING COUNT(DISTINCT o.mid) = COUNT(DISTINCT mei.mid)
        )

        -- Step 4: Select the guests who meet the criteria
        SELECT g.gid, g.tableid
        FROM Guest g
        JOIN GuestValid gv ON g.gid = gv.gid;
    \end{verbatim}
    \item Q3: Find the golden guest pair such that their total amount in two visits is maximum
    than any other guest pairs. Guest pairs are any two guests coming on the same day
    and occupy one table by the two.
    \begin{verbatim}
        WITH GuestVisitAmount AS (
            SELECT
                g1.gid AS gid1,
                g2.gid AS gid2,
                g1.tableid,
                o1.date,
                SUM(o1.quantity * m1.price) AS amount1,
                SUM(o2.quantity * m2.price) AS amount2,
                SUM(o1.quantity * m1.price) + SUM(o2.quantity * m2.price) 
                    AS total_amount
            FROM
                Guest g1
                JOIN Guest g2 ON g1.tableid = g2.tableid AND g1.gid < g2.gid
                JOIN Order o1 ON g1.gid = o1.gid
                JOIN Order o2 ON g2.gid = o2.gid AND o1.date = o2.date
                JOIN MenuItem m1 ON o1.mid = m1.mid
                JOIN MenuItem m2 ON o2.mid = m2.mid
            GROUP BY
                g1.gid, g2.gid, g1.tableid, o1.date
        ),

        MaxTotalAmount AS (
            SELECT
                gid1, gid2, total_amount
            FROM
                GuestVisitAmount
            ORDER BY
                total_amount DESC
            LIMIT 1
        )

        SELECT
            mta.gid1, mta.gid2, mta.total_amount
        FROM
            MaxTotalAmount mta;
    \end{verbatim}
\end{itemize}

\section*{Question 2}
For the queries generated from Q1, Q2, Q3.
\subsection*{(2-1)}
Argue whether each generated query is correctly expressing its query
description. Construct a table instance for explaining your answer. If the
generated query is wrong, point out its errors, and show a correct SQL query
that returns the expected answer for the query description.

\begin{itemize}
    \item For Q1, the generated query is correct. 
    \subitem \textbf{GuestOrderCount} table instance:  Calculate the number of times each guest has ordered each menu item. It groups the orders by `mid' and `gid' and counts the number of orders for each combination.
    \subitem \textbf{MenuItemGuestCount} table instance: Filter the results to include only those guests who ordered a menu item exactly once.
    \subitem The final part of the query selects the `mid' with the maximum count of guests who ordered the menu item exactly once.
    \item For Q2, the generated query is incorrect. The logic of the generated query is 
    to find the guests who have only ordered the most expensive item. However, guests who 
    always order an item that is most expensive within the table don't necessarily have to 
    order the most expensive item only. The correct query should be:
    \begin{verbatim}
        WITH GuestMaxPrice AS (
            SELECT 
                g.gid,
                g.tableid,
                o.date,
                MAX(m.price) AS max_price
            FROM 
                Guest g
                JOIN Order1 o ON g.gid = o.gid
                JOIN MenuItem m ON o.mid = m.mid
            GROUP BY 
                g.gid, g.tableid, o.date
        ),

        TableMaxPrice AS (
            SELECT 
                g.tableid,
                o.date,
                MAX(m.price) AS max_price
            FROM 
                Guest g
                JOIN Order1 o ON g.gid = o.gid
                JOIN MenuItem m ON o.mid = m.mid
            GROUP BY 
                g.tableid, o.date
        ),

        ConsistentMaxPriceGuests AS (
            SELECT 
                gmp.gid,
                gmp.tableid,
                gmp.date
            FROM 
                GuestMaxPrice gmp
                JOIN TableMaxPrice tmp ON 
                 gmp.tableid = tmp.tableid AND gmp.date = tmp.date
            WHERE 
                gmp.max_price = tmp.max_price
        ),

        EligibleGuests AS (
            SELECT 
                gid
            FROM 
                ConsistentMaxPriceGuests
            GROUP BY 
                gid
            HAVING 
                COUNT(*) = (SELECT COUNT(*) FROM GuestMaxPrice 
                    WHERE gid = ConsistentMaxPriceGuests.gid)
        )

        SELECT 
            g.gid,
            g.tableid
        FROM 
            Guest g
        JOIN 
            EligibleGuests eg ON g.gid = eg.gid;
    \end{verbatim}
    \item For Q3, the generated query is incorrect. The code only select the guest pair who 
    have the maximum total amount in one visit. 
    \begin{verbatim}
        WITH GuestVisitAmounts AS (
            SELECT 
                g.gid,
                g.tableid,
                o.date,
                SUM(o.quantity * m.price) AS total_amount
            FROM 
                Guest g
                JOIN Order1 o ON g.gid = o.gid
                JOIN MenuItem m ON o.mid = m.mid
            GROUP BY 
                g.gid, g.tableid, o.date
        ),

        GuestPairs AS (
            SELECT 
                gva1.gid AS gid1,
                gva2.gid AS gid2,
                gva1.tableid,
                gva1.date,
                gva1.total_amount AS amount1,
                gva2.total_amount AS amount2,
                gva1.total_amount + gva2.total_amount AS combined_amount
            FROM 
                GuestVisitAmounts gva1
                JOIN GuestVisitAmounts gva2 
                    ON gva1.tableid = gva2.tableid 
                    AND gva1.date = gva2.date 
                    AND gva1.gid < gva2.gid
        ),

        TotalAmountsByPair AS (
            SELECT 
                gid1, 
                gid2, 
                SUM(combined_amount) AS total_combined_amount
            FROM 
                GuestPairs
            GROUP BY 
                gid1, gid2
        ),

        MaxTotalAmountPair AS (
            SELECT 
                gid1, 
                gid2, 
                total_combined_amount
            FROM 
                TotalAmountsByPair
            ORDER BY 
                total_combined_amount DESC
            LIMIT 1
        )

        SELECT 
            mtap.gid1, 
            mtap.gid2, 
            mtap.total_combined_amount
        FROM 
            MaxTotalAmountPair mtap;
    \end{verbatim}
\end{itemize}

\subsection*{(2-2)}
Revise the natural language prompt so that the correct SQL query or its
equivalent query, for the query description is generated.
\begin{itemize}
    \item For Q2, the natural language prompt should be revised as follows:
    \subitem For the query "Find guests who always order an item that is most expensive within the table.", the targeted guest don't have to order the most expensive menu item. It should be his/her order's price is more expesive than other guest in the same table. And There isn't constraints on data, just table.
    \item For Q3, the natural language prompt should be revised as follows:
    \subitem For Query 3:"Find the golden guest pair such that their total amount in two visits is maximum
    than any other guest pairs. Guest pairs are any two guests coming on the same day
    and occupy one table by the two.", you need to find the guest pair that their total amount in two visits is maximum rather than comparing with one visit.
\end{itemize}

\end{document}